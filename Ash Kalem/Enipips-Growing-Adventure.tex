\documentclass[10pt,twoside,twocolumn,openany]{book}

% The following packages were required to make the dnd.sty work
\usepackage[table]{xcolor}
\usepackage{tikz}
\usepackage{everyshi}
\usepackage{keycommand}
\usepackage{fancyhdr}
\usepackage[most]{tcolorbox}
\usepackage{environ}
\usepackage{trimspaces}
\usepackage{fp}
\usepackage[pages=all]{background}
\usepackage{everypage}
\usepackage{listings}

\usepackage[bg-letter]{dnd} % Options: bg-a4, bg-letter, bg-full, bg-print, bg-none.

\usepackage[T1]{fontenc}
\usepackage[utf8]{inputenc} % this way umlaute are included from the get go
\usepackage[ngerman]{babel} % german spell check
\usepackage{lmodern}

\usepackage{graphicx}

\usepackage{datetime}
\usepackage{amssymb}

\usepackage{hyperref} % these two lines are so that the table of content is clickable
\hypersetup{linktoc=all}
\usepackage{tikz}
\usepackage[protrusion=true,expansion=true]{microtype}
\usetikzlibrary{positioning,shapes,shadows,arrows,backgrounds,fit}

%\setlength{\parindent}{0pt}

\definecolor{ultramarine}{RGB}{0,32,96}

\newcommand{\card}[4]{
	\begin{tikzpicture}[background rectangle/.style = {draw=black, fill=white, rounded corners}, show background rectangle, node distance=0.3cm]
	
	\node (Posessor) [textstyle, align=center, scale=1.5] {
		%\color{white}
		\large$\mathfrak{#1}$
	};
	\node (Obtained) [textstyle, below=of Posessor] {
		%\color{white}
		Obtained in Session #2.
	};
	\node (Knowledge) [textstyle, below=of Obtained, scale=1.3] {
		\begin{minipage}[t][3cm]{6cm}
		\ifthenelse{\equal{#3}{true}}{
			\color{ultramarine}
			#4
		}{
			\color{red}
			Content Hidden
		}
		\end{minipage}
	};	
	\end{tikzpicture}
}

\begin{document}
	\tikzstyle{textstyle}=[rectangle, text width=6cm, text badly ragged, scale=0.8]
	\begin{titlepage} % good tital page template, only needs the class X notes part to be changed for each new class
		
		\centering
		\includegraphics[width=1\textwidth]{AshkalemLogo}\par\vspace{1cm}
		{\scshape\LARGE Ashkalem Abenteuer \par}
		\vspace{1cm}
		{\huge\bfseries Die Zu Formende Gruppe \par}
		\vspace{2cm}
		{\Large\itshape by Enipip \par}
		
		\vfill
		
		% Bottom of the page
		
	\end{titlepage}
	
	\tableofcontents % creats a table of contents, ensured already that it is clickable
	\newpage % starts the actual document on a new page so there is no weird colision of text and toc
	
	\chapter{Backstories}
	\section{Der Entschlüssler}
	
	\newpage
	\section{Finnan}
	
	\newpage
	\section{Kodie Nimiel}
	
	\newpage
	\section{Enipip der Große}
	[Gnome]\newline[Aufgewachsen bei Menschen]\newline[möchte als ebenwürtig angesehen werden]\newline[Wurde im Waisenhaus von anderen Kindern gehänselt und verschlagen wegen seiner eigenen inneren Magie, wodurch er sich auch selbst etwas hasst]\newline[Hat beim Paladin tryout mit gemacht, hat aber nicht die richtige Magie auf Befehl beschwören]\newline[Hat sich eine General Rüstung aus Stoff nähen lassen, weil er das sein möchte um ein höheren Anblick von anderen zu erlangen]\newline[Als Kind wurde er mal so sehr gehänselt, dass er so wütend wurde, dass Geister seiner vorfahren ihm erschienen]\newline[Weil in Stadt von Paladinen aufgewachsen sieht er Magie, welche nicht von Paladinen kommt als Kezerei, leider auch seine eigene]\newline[Nachdem seine Eltern ihn verlassen hatten, hat ihn ein Paladin gefunden, der ihn ins Waisenhaus gebracht hat, aber ihn von mal zu mal besucht hat und als einziger Enipip nicht gehänselt hatte, wodurch Enipip zu ihm hoch geschaut hat]
	
	\chapter{World Knowledge}
	
	\card{Enipip}{1}{false}{Carmo lag oft mit den Templern im Krieg}
	
	
	\card{Enipip}{1}{false}{Extario ist ein Königreich Nördlich der Geistergraßsteppe. Regiert von König Darian dem Vasallen des Drachen Vindell Flammenhert.}
	
	
	\card{Enipip}{1}{false}{Herzsklaven sind Menschen die Liebhaber von Drachen wahren und sich einem Ritual unterzogen haben bei dem der Drache ihr Herz in seine Brust aufnimmt. Der Herzsklave stirbt nicht solange der Drache lebt, muss aber jedem seiner Befehle gehorchen.}
	
	\card{Der Entschlüssler}{1}{false}{Yet to be revealed}
	
	\card{Der Entschlüssler}{1}{false}{Yet to be revealed}
	
	\card{Der Entschlüssler}{1}{false}{Yet to be revealed}
	
	\card{Finnan}{1}{false}{Yet to be revealed}
	
	\card{Finnan}{1}{false}{Yet to be revealed}
	
	\card{Finnan}{1}{false}{Yet to be revealed}
	
	\card{Kodie}{1}{false}{Yet to be revealed}
	
	\chapter{Beschreibung}
	\section{Orte}
	\subsection{Tularon}
	\subsubsection{Besprechungsraum}
	\label{Besprechungsraum}
	Der Quadratische Raum ist stickig und der Rauch der Fackeln die Trotz des Warmenkristalllichts aufgestellt wurden helfen auch nicht gerade. Die Wände sind bedeckt von Lageplänen die Kristallvorkommen und Metallvenen zeigen, auf dem großen Tisch in der Mitte des Raumes steht ein kleines Modell eines Förderkrans.
	
	\subsection{Der Temerin}
	
	
	\subsubsection{Die Capricorn}
	\label{Capricorn}
	Die Capricorn ist ein sehr altes Schiff, was man daran erkennen kann, dass vom Material her, nichts vom Ursprungsboot zu gehören scheint. Das Deck besteht hauptsächlich aus Holzlatten, welche jegliche Löcher im Boden und in der Wand flicken. Das Segel hat ebenfalls bessere Tage mit erlebt. Auch das besteht aus großen Flicken und gibt dem Segel nicht nur den Anschein lange in gebrauch zu sein, sondern auch inzwischen eine Art Charakter entwickelt zu haben. \newline
	Die Crew besteht aus einem Kapitän und zwei Matrosen. 
	
	\subsection{Milbach}
	Nach der langen Fahrt lasst ihr die Geistergraßsteppe hinter euch und eine halbe stunde später seht ihr an einer Flussbiegung die Stadt Milbach deren Häuser sich an den Temerien Schmiegen. Doch von den Fachwerkhäusern steigt Rauch auf ein Viertel im Hafennähe steht lichterloh in Flammen Leute rennen mit Eimern in der Hand zum Fluss um den brand zu löschen. Auch jenseitz der Stadt sieht man lange schneisen verbrannten Graßes, wie als währe ein Rises mit einem Flammwerfer durch die Stadt gezogen.
	Der Kapitän beschattet seine Augen mit seiner Hand und murmelt:"Drachenfeuer...."
	
	\newpage
	\section{Jegliche begegnete Charaktere}
	\subsection{Kili Touleron}
	-DESCRIPTION-
	
	Eines Nachts, ohne etwas zu Ahnen, gingen alle Lichter in der Stadt aus und sorgten so für seinen Tod.
	
	\subsection{Der Onkel von Kili Touleron}
	\label{Onkel}
	-NAME- ist ein Zwerg der sich über die Tode in Tularon aufregt. Noch als Laichen aus der Höhle getragen wurden, suchte er nach Einer Gruppe, welche der Sache auf den Grund gehen kann. Die Haare die er noch hat schienen von seinem Kopf zu seinem Kinn gewandert zu sein. Beim ersten Antreffen ist in eine feine Rüstung gekleidet, mehr Repräsentativ als Praktisch. Außerdem scheint er gerne seine Pfeife zu rauchen.

	Sonst ist er ein netter, der gerne -DESCRIPTION-
	
	\subsection{Kapitän}
	Der Kapitän ist ein Halbling mit einem großen Schnauzer im Gesicht. Aus irgendeinem Grund scheint er voreingenommen, Gnomen gegenüber zu sein. 
	
	\chapter{Die Reise Beginnt}
	\section{Intro}
	3 Sekunden. 3 Sekunden braucht es um sich friedlich in sein Bett zu legen, 3 Sekunden braucht es die Lichter im Zimmer zu löschen, wohl wissend das die Leuchtenden Kristalle an den Wänden der Tularon Miene jegliche Dunkelheit von den fast 500 Einwohnern fernhalten würden.
	3 Sekunden, 3 Sekunden hat die Dunkelheit gebraucht um nach **Tularon** zu kommen, 3 Sekunden um jeden Mann jede Frau und jedes Kind das sich in sicherheit wähnte zu töten.*
	
	Unsere Geschichte beginnt an einem späten Nachmittag im südwesten der **Wornostberge**. Dort eingebettet in Grüne Tannen liegt die kleine Zwergenmiene Tularon. Noch vor 2 Tagen ein anlaufplatz für Händler die in der **Geistergrasssteppe** Handel treiben wollten ist sie heute ein lebendiges Grab.
	
	Und so beginnt unsere geschichte nicht mit einer Taverne voller gutgelaunter Bauern und Abenteurern die sich am ende ihres langen Tages ein Bier gönnen, sie beginnt nicht mit einem alten Mann in einer Ecke der den Abenteurern eine Quest anbietet die sie schwerlich ablehnen können.
	
	Nein unsere geschichte beginnt mit dem Gestank der Verwesung, sie beginnt mit dem dumpfen Umph mit dem Leichen auf Karren geladen werden, sie beginnt mit einem Onkel der Rache will, Rache für einen feigenn Angriff auf die Miene seines Neffen und Rache für alle Zwerge die letzte Nacht gestorben sind. Und so wendet sich dieser Onkel an euch, die fähigsten Abenteurer der Umgebung um die Schuldigen zu stellen und dafür zu sorgen das sich was auch immer hier geschehen ist niemals wiederholt.
	
	\section{Der Wunsch nach Rache}
	Auf der Anderen Seite des Tisches im \hyperref[Besprechungsraum]{Besprechungsraum} stand ein alter \hyperref[Onkel]{Zwerg}. Der Zorn stand ihm ins Gesicht geschrieben: \newline
	"Ich will das ihr diese Bastarde findet die das getan haben findet sie und hängt sie an ihren eigenen Gendärmen auf!!!" 
	
	Etwas Zeit verging, in der sich der Onkel weiter über das Geschehen aufregte. Nach seinem Rant erklärte er uns, dass unser erstes Ziel der Hafen (Itimon?) sein werde, wo ein Schiff auf uns warte, welches uns nach Milbach bringen würde. 
	
	Nach der Besprechung gingen wir alle raus und bereiteten uns auf unsere Reise vor. Der Entschlüssler wollte sich nochmal die Laichen anschauen, um zu sehen, ob er einen Ansatz entwickeln könnte zu was passiert haben könnte. Doch dies geschah ohne Erfolg. \newline
	Etwas später machten wir uns dann auf den Weg zum Hafen. Einmal da angekommen, erwartete uns die Capricorn
	
	Der Anblick des Schiffs lies sich zu wünschen, doch der Kapitän versicherte uns, dass es uns sicher an unser Ziel bringen würde. Vor der Abreise kommandierte der Kapitän seine 2 Matrosen noch herum, um das Schiff für die Fahrt bereit zu machen. 
	
	\section{Auf Hohem Fluss}
	Vor euren Augen weitet sich der Wald und ihr seht ein verladedock jenseits dessen sich ein Fluss erstreckt der sich vor Vergleichen mit dem Nil nicht zu scheuen braucht: der Temerin.
	Besonders Fällt euch sein Wasser auf das eine bläulich grüne Farbe hat und leicht zu leuchten scheint durch die Pollen des Geistergraßes dier auf seinem Weg durch die Steppe aufsammelt
\end{document}